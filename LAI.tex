\documentclass[12pt,a4paper]{article}
\usepackage[utf8]{inputenc}
\usepackage[german]{babel}
\usepackage[T1]{fontenc}
\usepackage{amsmath}
\usepackage{amsfonts}
\usepackage{amssymb}
\usepackage{makeidx}
\usepackage{graphicx}
\author{Heftige Hausaufgabengruppe}
\title{Lineare Algebra Für Informatiker}
\begin{document}
\maketitle
\tableofcontents
\pagebreak
\section{Generell Wissenswertes}
Wir haben eine Mail von der Übungsleiterin bekommen, aus der hervorgeht, dass (nur) Kapitel 7 bis Kapitel 12 wirklich Prüfungsrelevant sein werden:
\begin{quote}
Ansonsten wird sich das Tutorium hauptsächlich an den Übungsblättern
7-12 orientieren.
(Klausur wird fast ausschließlich aus dieser Thematik bestehen)
\end{quote}
Deshalb wäre es wahrscheinlich am sinnvollsten zu aller erst die Themen dieser Übungsblätter zu lernen. Wenn danach noch Zeit ist kann man sich ja den anderen Themen auch noch widmen. Übrigens wird im Moodle mittlerweile zu jedem der Übungsblätter eine Lösung angeboten, was das Lernen zusätzlich vereinfachen sollte.
\end{document}